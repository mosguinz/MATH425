\documentclass[12pt]{article}
\usepackage[margin=2.5cm]{geometry}
\usepackage{fancyhdr,latexsym,amssymb,amsmath,graphicx}
\usepackage{algorithm}
\usepackage{cancel}
\usepackage[noend]{algpseudocode}
\usepackage[pdftex]{hyperref}
\pagestyle{fancy}
\usepackage[parfill]{parskip} % Do not indent between empty lines
\usepackage{mathtools}
\usepackage{nicematrix}
\usepackage{xfrac}
\usepackage{bm}
\usepackage{esvect}

% For the \set notation
\usepackage{xparse} 
\DeclarePairedDelimiterX{\set}[1]{\{}{\}}{\setargs{#1}}
\NewDocumentCommand{\setargs}{>{\SplitArgument{1}{;}}m}
{\setargsaux#1}
\NewDocumentCommand{\setargsaux}{mm}
{\IfNoValueTF{#2}{#1} {#1\,\delimsize|\,\mathopen{}#2}}%{#1\:;\:#2}

% Some common symbols
\newcommand{\veq}{\mathrel{\rotatebox{90}{$=$}}}
\newcommand{\vneq}{\mathrel{\rotatebox{90}{$\neq$}}}
\newcommand{\vect}[1]{\vv{\mathbf{#1}}}
\newcommand{\code}[1]{\texttt{#1}}
\newcommand{\R}{\mathbb{R}}
\newcommand{\rank}{\operatorname{rank}}

% \lhead{MATH 425}
\lhead{Computation of the Camera Matrix}
\rhead{Sitthisarnwattanachai}

\title{Computation of the Camera Matrix}
\author{mosguinz}
\date{December 2024}

\begin{document}

\section{Introduction}

In computer vision, the camera matrix is a $3 \times 4$ matrix that encapsulates the mapping performed by a pinhole camera, projecting 3D points in the real world onto 2D points in an image. This ``mini"-project focuses on understanding and implementing the simplest camera model, the basic pinhole camera.

We will begin by describing the projection of points in 3D space onto a 2D plane using a series of matrices that we will define. The majority of this report is based on the explanations provided in \textit{Multiple View Geometry in Computer Vision} by Richard Hartley and Andrew Zisserman. Finally, we will discuss the implementation of a specific algorithm in MATLAB to compute the camera matrix.

\subsection{TODO: Notational note}

To remain consistent with the textbook by Hartley and Zisserman, we will keep our notations largely similar. To emphasize a couple of matrices, I am going to 

\section{The Pinhole Camera Model}

The pinhole camera model is a simplified representation of how a camera projects three-dimensional (3D) points in space onto a two-dimensional (2D) image plane. Under this model, we aim to map a 3D point, denoted by 
$\mathbf{X} = \begin{pmatrix} x & y & z \end{pmatrix}^\top$, 
to its corresponding point on the image plane. This point is the intersection of the line passing through $\mathbf{X}$ and the center of projection (the camera's optical center) with the image plane.

The geometry of the setup defines the center of projection at the origin of the camera coordinate system, with the image plane located at a fixed distance $f$ (the focal length) from the origin. The projection follows the principle of similar triangles.

[[TODO: INSERT DIAGRAM HERE]]

The equations governing this projection are given by:
$$
u = \frac{f x}{z}, \quad v = \frac{f y}{z}.
$$

Now, note that these perspective projections are nonlinear due to the division by $z$. To simplify our operations, we want to transform our system using a linear model. In our case, we can accomplish this by using homogeneous coordinates.

\subsection{Linearizing and using homogenous coordinates}

In the homogeneous form, the coordinates of our point $\mathbf{X}$ are expressed as $\begin{pmatrix} x & y & z & 1 \end{pmatrix}^\top$, and the corresponding point on the image plane becomes $\begin{pmatrix} u & v & z \end{pmatrix}^\top$. Then, we can write our projection in matrix form as:
$$
\begin{pmatrix} u \\ v \\ z \end{pmatrix} = 
\begin{pmatrix}
    f & 0 & 0 & 0 \\
    0 & f & 0 & 0 \\
    0 & 0 & 1 & 0
\end{pmatrix}
\begin{pmatrix} x \\ y \\ z \\ 1 \end{pmatrix}.
$$

This homogeneous representation serves as the foundation for deriving the complete camera matrix, which incorporates additional parameters to account for intrinsic and extrinsic camera properties.

\subsection{Intrinsic Matrix}

In real-world cameras, the optical center does not necessarily coincide with the center of the image plane. Instead, the projection may be offset by a certain amount. This offset is characterized by the principal point, denoted by $p_x$ and $p_y$ for the horizontal and vertical displacements, respectively.

Thus, the projection equations are modified to account for this offset:
$$
u = \frac{f x}{z} + p_x, \quad v = \frac{f y}{z} + p_y.
$$

In the homogeneous representation, this is expressed as:
$$
\begin{pmatrix} u \\ v \\ z \end{pmatrix} = 
\begin{pmatrix}
f & 0 & p_x & 0 \\
0 & f & p_y & 0 \\
0 & 0 & 1 & 0
\end{pmatrix}
\begin{pmatrix} x \\ y \\ z \\ 1 \end{pmatrix}.
$$

Here, the $3 \times 3$ submatrix
$$
\mathbf{K} = \begin{pmatrix}
f & 0 & p_x \\
0 & f & p_y \\
0 & 0 & 1
\end{pmatrix}
$$
is known as the \textit{camera calibration matrix}, which encapsulates the \textit{intrinsic parameters} of the camera, such as the focal length $f$ and the principal point offsets, $p_x$ and $p_y$.

The entire matrix is referred to as the \textit{intrinsic matrix}. Now, our 2D coordinate  $\mathbf{x}$ can be expressed as
$$
\mathbf{x} = \mathbf{K}\begin{bmatrix}
    I & | & \vect{0}
\end{bmatrix}\mathbf{x}_\text{cam}
$$
where $\mathbf{x}_\text{cam}=\begin{pmatrix}
    x & y & z & 1
\end{pmatrix}^\top$ is our camera at the origin of a Euclidean coordinate system.

\subsection{Extrinsic Matrix}

In addition to the intrinsic properties of a camera, we must account for its position and orientation in the world. These are described by the \textit{extrinsic parameters}, which define the relationship between the \textit{camera coordinate frame} (CCF) and the \textit{world coordinate frame} (WCF). 

The transformation between the WCF and the CCF involves two components:
\begin{itemize}
    \item A $3 \times 3$ rotation matrix $\mathbf{R}$, which describes the orientation of the camera with respect to the WCF. Note that this matrix is orthogonal.
    \item A $3 \times 1$ translation vector $\mathbf{t}$, which specifies the position of the camera's optical center in the WCF.
\end{itemize}

The matrix $\mathbf{R}$ has three rows, each representing a basis vector of the camera coordinate frame expressed in the world coordinate frame:
\begin{itemize}
    \item The first row of $\mathbf{R}$ represents the direction of the camera's $x$-axis in the WCF.
    \item The second row of $\mathbf{R}$ represents the direction of the camera's $y$-axis in the WCF.
    \item The third row of $\mathbf{R}$ represents the direction of the camera's optical axis (or $z$-axis) in the WCF.
\end{itemize}

We again express the system in terms of a homogeneous coordinate to transform a point $\mathbf{X}_{\text{w}}$ in the WCF to the CCF, we apply the rotation and translation as follows:
$$
\mathbf{x}_\text{cam} = 
\mathbf{R} \mathbf{X}_{\text{w}} + \mathbf{t} =
\begin{pmatrix} 
\mathbf{R} & \mathbf{t} \\
0 & 1
\end{pmatrix}
\begin{pmatrix} x_{\text{w}} \\ y_{\text{w}} \\ z_{\text{w}} \\ 1 \end{pmatrix}.
$$

Here, the $4\times 4$ matrix $
\begin{pmatrix} 
\mathbf{R} & \mathbf{t} \\
0 & 1
\end{pmatrix}
$
is known as the \textit{extrinsic matrix}, combining rotation and translation into a single transformation.

\section{The Camera Matrix}

The camera matrix combines the intrinsic and extrinsic transformations to map a 3D point in the world coordinate frame directly onto the 2D image plane.

The intrinsic matrix $\mathbf{K}$ maps a point from the camera coordinate system to the 2D image plane. The extrinsic matrix $\begin{bmatrix} \mathbf{R} & | & \mathbf{t} \end{bmatrix}$ transforms a point from the world coordinate frame to the camera coordinate frame, where the rotation matrix $\mathbf{R}$ encodes the orientation of the camera and the translation vector $\mathbf{t}$ specifies its position.

Together, the intrinsic and extrinsic matrices form our $3\times 4$ camera matrix, $\mathbf{P}$:
\begin{align*}
    \mathbf{P} &= \mathbf{K} \begin{bmatrix} \mathbf{R} & | & \mathbf{t} \end{bmatrix} \\
    &= \begin{pmatrix}
        f & 0 & 0 & 0 \\
        0 & f & 0 & 0 \\
        0 & 0 & 1 & 0
    \end{pmatrix}
    \begin{pmatrix}
        r_{11} & r_{12} & r_{13} & t_x \\
        r_{21} & r_{22} & r_{23} & t_y \\
        r_{31} & r_{32} & r_{33} & t_z \\
        0 & 0 & 0 & 1
    \end{pmatrix}.
\end{align*}

The transformation can now be expressed compactly as:
$$
\mathbf{x}_{\text{image}} = \mathbf{P} \mathbf{X}_{\text{w}},
$$
where $\mathbf{X}_{\text{w}}$ is a 3D point in homogeneous world coordinates, and $\mathbf{x}_{\text{image}}$ is its corresponding 2D point in homogeneous image coordinates.

Now that we have a complete overview of what the camera matrix \( \mathbf{P} \) is composed of, we can compute an initial estimate for \( \mathbf{P} \) using the Gold Standard algorithm outlined in the textbook (Hartley and Zisserman 181). The remainder of this report will be dedicated to computing each of the following steps.

\subsection{Steps of the Gold Standard Algorithm}

\paragraph{(1) Normalization}
We first normalize both the image points and the 3D space points:
\begin{itemize}
    \item Apply a similarity transformation \( \mathbf{T} \) to normalize the image points. This involves translating and scaling the points so that their centroid is at the origin and their average distance from the origin is \(\sqrt{2}\).
    \item Similarly, apply a similarity transformation \( \mathbf{U} \) to normalize the 3D space points so that their centroid is at the origin, and their average distance from the origin is \(\sqrt{3}\).
\end{itemize}

\paragraph{(2) Direct Linear Transformation (DLT)}
The next step is to form a linear system of equations:
\begin{itemize}
    \item For each correspondence \( \mathbf{X}_i \leftrightarrow \mathbf{x}_i \), use the normalized coordinates \( \tilde{\mathbf{X}}_i \) and \( \tilde{\mathbf{x}}_i \) to generate two equations based on the projection model \( \tilde{\mathbf{x}}_i = \mathbf{P} \tilde{\mathbf{X}}_i \).
    \item Stack all the equations into a \( 2n \times 12 \) matrix \( \mathbf{A} \), where \( n \) is the number of correspondences.
    \item Write the entries of \( \mathbf{P} \) as a 12-dimensional vector \( \mathbf{p} \). The system \( \mathbf{A} \mathbf{p} = 0 \) is then solved subject to the constraint \( \| \mathbf{p} \| = 1 \).
    \item The solution is obtained from the unit singular vector of \( \mathbf{A} \) corresponding to its smallest singular value.
\end{itemize}

\paragraph{(3) Denormalization}
Finally, denormalize the estimated \( \mathbf{P} \) to obtain the camera matrix in the original coordinate system:
$$
\mathbf{P} = \mathbf{T}^{-1} \tilde{\mathbf{P}} \mathbf{U}.
$$

Here, \( \mathbf{T} \) and \( \mathbf{U} \) are the inverse similarity transformations used for normalization, and \( \tilde{\mathbf{P}} \) is the intermediate result obtained from the DLT step.

\section{Normalization}

Normalization is a crucial preprocessing step in the Gold Standard Algorithm, designed to improve numerical stability when estimating the camera matrix. By scaling and centering the points appropriately, we mitigate the effects of numerical inaccuracies in the subsequent computations.

\textit{Data normalization is an essential step in the DLT algorithm and must not be considered optional} (ibid. 108).

\subsection{Normalizing Image Points}
Given a set of image points \( \mathbf{x}_i = \begin{pmatrix} u_i & v_i \end{pmatrix}^\top \) in homogeneous coordinates, the goal is to apply a similarity transformation \( \mathbf{T} \) such that:
\begin{itemize}
    \item The centroid of the points is shifted to the origin.
    \item The average distance of the points from the origin is \(\sqrt{2}\).
\end{itemize}

To achieve this, the transformation matrix \( \mathbf{T} \) is defined as:
$$
\mathbf{T} = 
\begin{pmatrix} 
s & 0 & -s \bar{u} \\
0 & s & -s \bar{v} \\
0 & 0 & 1 
\end{pmatrix},
$$
where:
\begin{itemize}
    \item \( \bar{u}, \bar{v} \) are the centroid coordinates of the points, computed as:
    $$
    \bar{u} = \frac{1}{n} \sum_{i=1}^n u_i, \quad \bar{v} = \frac{1}{n} \sum_{i=1}^n v_i.
    $$
    \item \( s \) is the scaling factor, defined to ensure the average distance from the origin is \(\sqrt{2}\):
    $$
    s = \frac{\sqrt{2}}{\displaystyle\frac{1}{n} \sum_{i=1}^n \sqrt{(u_i - \bar{u})^2 + (v_i - \bar{v})^2}}.
    $$
\end{itemize}

\subsection{Normalizing 3D Space Points}
Similarly, for a set of 3D space points \( \mathbf{X}_i = \begin{pmatrix} x_i & y_i & z_i \end{pmatrix}^\top \) in homogeneous coordinates, we apply a similarity transformation \( \mathbf{U} \) such that:
\begin{itemize}
    \item The centroid of the points is shifted to the origin.
    \item The average distance of the points from the origin is \(\sqrt{3}\).
\end{itemize}

The transformation matrix \( \mathbf{U} \) is defined as:
$$
\mathbf{U} = 
\begin{pmatrix} 
s_x & 0 & 0 & -s_x \bar{x} \\
0 & s_y & 0 & -s_y \bar{y} \\
0 & 0 & s_z & -s_z \bar{z} \\
0 & 0 & 0 & 1
\end{pmatrix},
$$
where:
\begin{itemize}
    \item \( \bar{x}, \bar{y}, \bar{z} \) are the centroid coordinates of the 3D points, computed as:
    $$
    \bar{x} = \frac{1}{n} \sum_{i=1}^n x_i, \quad \bar{y} = \frac{1}{n} \sum_{i=1}^n y_i, \quad \bar{z} = \frac{1}{n} \sum_{i=1}^n z_i.
    $$
    \item \( s_x, s_y, s_z \) are the scaling factors for each axis, defined to ensure the average distance from the origin is \(\sqrt{3}\):
    $$
    s_x = s_y = s_z = \frac{\sqrt{3}}{\displaystyle\frac{1}{n} \sum_{i=1}^n \sqrt{(x_i - \bar{x})^2 + (y_i - \bar{y})^2 + (z_i - \bar{z})^2}}.
    $$
\end{itemize}

After normalization, we have that:
\begin{itemize}
    \item the image points \( \mathbf{x}_i \) are transformed into normalized coordinates \( \tilde{\mathbf{x}}_i = \mathbf{T} \mathbf{x}_i \); and
    \item the 3D space points \( \mathbf{X}_i \) are transformed into normalized coordinates \( \tilde{\mathbf{X}}_i = \mathbf{U} \mathbf{X}_i \).
\end{itemize}
The transformations \( \mathbf{T} \) and \( \mathbf{U} \) ensure that the numerical conditioning of the problem is improved, making the DLT step more stable and accurate.


\end{document}
