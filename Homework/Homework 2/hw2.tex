\documentclass[11pt]{amsart}

\usepackage{amsthm, amssymb,amsmath}
\usepackage{graphicx}

\theoremstyle{definition}  % Heading is bold, text is roman
\newtheorem{theorem}{Theorem}
\newtheorem{definition}{Definition}
\newtheorem{example}{Example}

\newcommand{\ojo}[1]{{\sffamily\bfseries\boldmath[#1]}}

\oddsidemargin 0pt
\evensidemargin 0pt
\marginparwidth 0pt
\marginparsep 10pt
\topmargin -10pt
\headsep 10pt
\textheight 8.4in
\textwidth 7in

%\input{../header}


\begin{document}

%\homework{}{Homework II}


\noindent For this homework, include all code and computations in a {\tt MATLAB} file named {\tt math425hw2.m}.
You will need to submit this file along with a document containing your answers which do not
involve {\tt MATLAB}.\\


\noindent
{\bf 1.a)} Create a function called {\tt myPartialPivot} which takes as input an $n \times n$ matrix $A$. The output is an $n \times n$
matrix $U$ which is upper triangular.  This time we are not assuming that $A$ is regular, i.e., along the way some pivots could be zero. Even if a pivot
is not zero, use partial pivoting to identify a better pivot to continue with Gaussian elimination. \\
{\bf b)} Create a function called {\tt myRank} which takes as input an $n \times n$ matrix $A$ and computes the rank of $A$ using {\tt myPartialPivot}.\\
{\bf c)}  To test your function {\tt myRank}, generate a random $5 \times 3$ matrix $P$ and a random $3 \times 5$ matrix $Q$. Let $A = PQ$. Does your function
compute the rank of $A$ to be $3$?\\

\noindent {\bf 2.} Suppose an $n \times n$ matrix $A$ is {\it strictly column diagonally dominant}. This means that for each $j=1,\ldots, n$
$$ |a_{jj} | \, > \, \sum_{i \neq j}^n |a_{ij}|. $$ 
{\bf a)} Give an example of a $ 4 \times 4$ strictly column diagonally dominant matrix which is not a diagonal matrix. \\
{\bf b)} Show that if Gaussian elimination with partial pivoting is used on a strictly column diagonally dominant matrix no row interchanges occur. \\
{\bf c)} Modify {\tt myPartialPivot}  slightly so that it counts and prints the number of row interchanges during the application of the function.  Then 
test this function on your example from part {\bf a)}. The number of row interchanges should be zero. \\

\noindent {\bf 3.} Let $A$ be an $n \times n$ symmetric matrix. Describe a strategy of {\it symmetric pivoting} so that after the Gaussian elimination the matrix
$A$ is reduced to a diagonal matrix $D$. Make sure to argue that after each symmetric pivoting the resulting matrix is still symmetric. [Hint: this will require both row and
column operations]\\


 


\end{document}


