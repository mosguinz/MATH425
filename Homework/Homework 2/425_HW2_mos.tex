\documentclass[12pt]{article}
\usepackage[margin=2.5cm]{geometry}
\usepackage{fancyhdr,latexsym,amssymb,amsmath,graphicx}
\usepackage{algorithm}
\usepackage{cancel}
\usepackage[noend]{algpseudocode}
\usepackage[pdftex]{hyperref}
\pagestyle{fancy}
\usepackage[parfill]{parskip} % Do not indent between empty lines
\usepackage{mathtools}
\usepackage{nicematrix}
\usepackage{xfrac}
\usepackage{bm}
\usepackage{esvect}

\newcommand{\veq}{\mathrel{\rotatebox{90}{$=$}}}
\newcommand{\vneq}{\mathrel{\rotatebox{90}{$\neq$}}}
\newcommand{\vect}[1]{\vv{\mathbf{#1}}}
\newcommand{\code}[1]{\texttt{#1}}

\lhead{MATH 425}
\chead{Homework 1}
\rhead{Sitthisarnwattanachai}

\title{510 HW1}
\author{mosguinz}
\date{February 2024}

\begin{document}

\section*{Question 1}

The functions are defined in the \code{math425hw2.m} file, under the section \code{\%\% Question 1}.

\subsection*{Part A}

See \code{math425hw2.m}.

\subsection*{Part B}

Using \code{myPartialPivot}, \code{myRank} simply counts the number of non-zero pivots in the upper triangular matrix returned from \code{myPartialPivot}.

Note that, even with partial pivoting, the computation still suffer from the precision issue of floating-point arithmetic. For this reason, in the implementation for \code{myRank}, I have rounded pivot entries to the tenth decimal place to determine whether they are really zeroes.

For example, consider a matrix
$$
A=\begin{pmatrix}
    1 & 2 & 3 \\ 4 & 5 & 6 \\ 7 & 8 & 9
\end{pmatrix}.
$$

For the first pivot, we would row swap $R_3$ and $R_1$ and perform the row eliminations as follows.

$$
\begin{pmatrix}
     7 & 8 & 9 \\
     4 & 5 & 6 \\
     1 & 2 & 3
\end{pmatrix}
\xrightarrow[]{-\frac{4}{7}R_1 + R_2 \to R_2}
\begin{pmatrix}
     7 & 8 & 9 \\
     0 & \sfrac{3}{7} & \sfrac{6}{7} \\
     1 & 2 & 3
\end{pmatrix}
\xrightarrow[]{-\frac{1}{7}R_1 + R_3 \to R_3}
\begin{pmatrix}
     7 & 8 & 9 \\
     0 & \sfrac{3}{7} & \sfrac{6}{7} \\
     0 & \sfrac{6}{7} & \sfrac{12}{7}
\end{pmatrix}
$$

Then, for the second pivot, we perform the row operation $-\frac{1}{2}R_2 + R_3 \to R_3$. This also happens to take out the third pivot. Thus, the rank is 2.

$$
\begin{pmatrix}
     7 & 8 & 9 \\
     0 & \sfrac{6}{7} & \sfrac{12}{7} \\
     0 & \sfrac{3}{7} & \sfrac{6}{7}
\end{pmatrix}
\xrightarrow[]{-\frac{1}{2}R_2 + R_3 \to R_3}
\begin{pmatrix}
     7 & 8 & 9 \\
     0 & \sfrac{6}{7} & \sfrac{12}{7} \\
     0 & 0 & 0
\end{pmatrix}
$$

The above should be our upper triangular matrix, $U$. However, due floating point error, the result is not quiet zero. In MATLAB, when we display $U$ using \code{format rational}, it displays:

\begin{verbatim}
       7              8              9       
       0              6/7           12/7     
       0              *              *   
\end{verbatim}

where the asterisks denote that the denominator is too large. By using \code{format long} we can see that the result is pretty much zero:

\begin{verbatim}
    
   7.000000000000000   8.000000000000000   9.000000000000000
                   0   0.857142857142857   1.714285714285714
                   0   0.000000000000000   0.000000000000000
\end{verbatim}

Using \code{format longG} we can see that the stored values aren't quiet zero. Hence, I have rounded the digits to ten decimal places.

\begin{verbatim}
       7                         8                         9
       0         0.857142857142857          1.71428571428571
       0      5.55111512312578e-17      1.11022302462516e-16
\end{verbatim}

\subsection*{Part C}

Yes, the function virtually always returns 3. Since the question did not specify the bounds for the values from which a matrix could be generated, I simply used \code{rand(5, 3)} and \code{rand(3, 5)} to generate a $5\times3$ and $3\times5$ matrix, respectively. As such, it is virtually impossible for two or more rows to be identical or be a multiple of each other.

Furthermore, the floating-point precision issue discussed in part (b) would almost certainly ensure that the result appears to be full rank --- especially when the values in the matrix are floating-point numbers to begin with --- as values that should be zero may instead be interpreted as non-zero.

\end{document}
