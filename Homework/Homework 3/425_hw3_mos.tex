\documentclass[12pt]{article}
\usepackage[margin=2.5cm]{geometry}
\usepackage{fancyhdr,latexsym,amssymb,amsmath,graphicx}
\usepackage{algorithm}
\usepackage{cancel}
\usepackage[noend]{algpseudocode}
\usepackage[pdftex]{hyperref}
\pagestyle{fancy}
\usepackage[parfill]{parskip} % Do not indent between empty lines
\usepackage{mathtools}
\usepackage{nicematrix}
\usepackage{xfrac}
\usepackage{bm}
\usepackage{esvect}

% For the \set notation
\usepackage{xparse} 
\DeclarePairedDelimiterX{\set}[1]{\{}{\}}{\setargs{#1}}
\NewDocumentCommand{\setargs}{>{\SplitArgument{1}{;}}m}
{\setargsaux#1}
\NewDocumentCommand{\setargsaux}{mm}
{\IfNoValueTF{#2}{#1} {#1\,\delimsize|\,\mathopen{}#2}}%{#1\:;\:#2}

% Some common symbols
\newcommand{\veq}{\mathrel{\rotatebox{90}{$=$}}}
\newcommand{\vneq}{\mathrel{\rotatebox{90}{$\neq$}}}
\newcommand{\vect}[1]{\vv{\mathbf{#1}}}
\newcommand{\code}[1]{\texttt{#1}}
\newcommand{\R}{\mathbb{R}}
\newcommand{\rank}{\operatorname{rank}}

\lhead{MATH 425}
\chead{Homework 3}
\rhead{Sitthisarnwattanachai}

\title{MATH 425 Homework 3}
\author{mosguinz}
\date{October 2024}

\begin{document}

\section*{Question 1}

\subsection*{Part A}

Note that the question statement is not true if either $\vect{v}$ or $\vect{m}$ are zero vectors. The resulting $m\times n$ matrix would be a zero matrix, and thus its rank would be $0$.

For nonzero vectors $\vect{v}\in\R^m$ and $\vect{w}\in\R^n$, we have that:
$$
\vect{v}\vect{w}^\top = \begin{pmatrix}
    v_1 \\ v_2 \\ \vdots \\ v_m
\end{pmatrix}
\begin{pmatrix}
    w_1 & w_2 & \cdots & w_n
\end{pmatrix}
= \begin{pmatrix}
    v_1w_1 & v_1w_2 & \cdots & v_1w_n \\
    v_2w_1 & v_2w_2 & \cdots & v_2w_n \\
    \vdots & \vdots & \ddots & \vdots \\
    v_mw_1 & v_mw_2 & \cdots & v_mw_n
\end{pmatrix}
$$

Intuitively, we can already see that each row is being multiplied by the first column vector $\vect{v}$. So of course, the each column in $\vect{v}\vect{w}^\top$ is a multiple of $\vect{v}$. Or, we can perform the usual Gaussian elimination step. First, we can eliminate the second row by applying the following row operation:
$$
\begin{pmatrix}
    v_1w_1 & v_1w_2 & \cdots & v_1w_n \\
    v_2w_1 & v_2w_2 & \cdots & v_2w_n \\
    \vdots & \vdots & \ddots & \vdots \\
    v_mw_1 & v_mw_2 & \cdots & v_mw_n
\end{pmatrix}
\xrightarrow{-\frac{v_2}{v_1}R_1+R_2 \to R_2}
\begin{pmatrix}
    v_1w_1 & v_1w_2 & \cdots & v_1w_n \\
    0 & 0 & \cdots & 0 \\
    \vdots & \vdots & \ddots & \vdots \\
    v_mw_1 & v_mw_2 & \cdots & v_mw_n
\end{pmatrix}
$$

As such, we can apply the row operation $-\frac{v_i}{v_1}R_1 + R_i \to R_i$ for all $i=2,3,\ldots,m$ to eliminate all the rows below. Thus,
$$
\rank\begin{pmatrix}
    v_1w_1 & v_1w_2 & \cdots & v_1w_n \\
    0 & 0 & \cdots & 0 \\
    \vdots & \vdots & \ddots & \vdots \\
    0 & 0 & \cdots & 0 \\
\end{pmatrix} = \rank(\vect{v}\vect{w}^\top) = 1.
$$

\subsection*{Part B}

Let $A$ be an $m\times n$ matrix where
$$
A = \begin{pmatrix}
    a_{1,1} & a_{1,2} & \cdots & a_{1,n} \\
    a_{2,1} & a_{2,2} & \cdots & a_{2,n} \\
    \vdots & \vdots & \ddots & \vdots \\
    a_{m,1} & a_{m,2} & \cdots & a_{m,n}
\end{pmatrix}.
$$

Let $R_i$ denote the rows $i=1,\ldots,m$ in $A$. For example, $R_1 = \begin{pmatrix}
    a_{1,1} & a_{1,2} & \cdots & a_{1,n}
\end{pmatrix}$. Since $\rank(A)=1$, we know that all rows in $A$ are a multiple of each other. As such, we can write the matrix $A$ as
$$
A = \begin{pmatrix}
    R_1 \\ R_2 \\ \vdots \\ R_m
\end{pmatrix}
=
\begin{pmatrix}
    c_1R_1 \\ c_2R_1 \\ \vdots \\ c_mR_1
\end{pmatrix}
$$
for some scalars $c_1,c_2,\ldots,c_m\in\R$.

Then, let $\vect{v}\in\R^m$ be a column vector formed by the scalars $c_1,c_2,\ldots,c_m\in\R$:
$$
\vect{v}=\begin{pmatrix}
    c_1 \\ c_2 \\ \vdots \\ c_m
\end{pmatrix}
$$
and let $\vect{w}\in\R^n$ be a column vector comprised of the elements in the first row:
$$
\vect{w} = R_1^\top = \begin{pmatrix}
    a_{1,1} \\ a_{1,2} \\ \vdots \\ a_{1,n}
\end{pmatrix}
$$.
And so, we have that:
$$
    A = \vect{v}\vect{w}^\top 
    = \begin{pmatrix}
        c_1 \\ c_2 \\ \vdots \\ c_m
    \end{pmatrix}
    \begin{pmatrix}
        a_{1,1}  a_{1,2} & \cdots & a_{1,n}
    \end{pmatrix}
    = \begin{pmatrix}
        c_1a_{1,1} & c_1a_{1,2} & \cdots & c_1a_{1,n} \\
        c_2a_{1,1} & c_2a_{1,2} & \cdots & c_2a_{1,n} \\
        \vdots & \vdots & \ddots & \vdots \\
        c_ma_{1,1} & c_ma_{1,2} & \cdots & c_ma_{1,n} \\
    \end{pmatrix},
$$
which is precisely an $m\times n$ matrix with rank one. We can verify this by performing Gaussian elimination in the manner described in 1(a). Again, this holds for $\vect{v},\vect{w}\neq\vect{0}$.

\section*{Question 2}

Refer to Section \code{\%\% Question 2} in the \code{math425hw3.m} file for the relevant code.

Using MATLAB, we can simply augment the vectors into a matrix and use \code{rref} to find its reduced-row echelon form.

$$
\operatorname{rref}\begin{pNiceArray}{ccc|c}
    1 & 0 & 2 & 3 \\
    2 & -1 & 0 & 0 \\
    0 & 3 & 1 & -1 \\
    1 & 0 & -1 & 2
\end{pNiceArray}
= \begin{pNiceArray}{ccc|c}
    1 & 0 & 0 & 0 \\
    0 & 1 & 0 & 0 \\
    0 & 0 & 1 & 0 \\
    0 & 0 & 0 & 1 \\
\end{pNiceArray}
$$

Since the last row produces an inconsistent solution i.e., $0=1$, the vector $\begin{pmatrix}
    3 \\ 0 \\ -1 \\ 2
\end{pmatrix}$ is not a linear combination of the other three vectors. That is, there exists no $x_1,x_2,x_3\in\R$ such that $x_1\begin{pmatrix}
    1 \\ 2 \\ 0 \\ 1
\end{pmatrix} + x_2\begin{pmatrix}
    0 \\ -1 \\ 3 \\ 0
\end{pmatrix} + x_3\begin{pmatrix}
    2 \\ 0 \\ 1 \\ -1
\end{pmatrix} = \begin{pmatrix}
    3 \\ 0 \\ -1 \\ 2
\end{pmatrix}$.

\section*{Question 3}

For this question, let 
$$
{\vect v}_1 = \begin{pmatrix} 1 \\ 0 \\ 2  \end{pmatrix}, 
{\vect v}_2 = \begin{pmatrix} 3 \\ -1 \\ 1  \end{pmatrix},
{\vect v}_3 = \begin{pmatrix} 2 \\ -1 \\ -1  \end{pmatrix}\text{, and}\,
{\vect v}_4 = \begin{pmatrix} 4 \\ -1\\ 3  \end{pmatrix}.
$$

\subsection*{Part A}

Using MATLAB, we can augment the vectors and use the \code{rank} function to ascertain the number of linearly independent vectors. To span $\R^3$, we need three linearly independent vectors in $\R^3$. Since
$$
\rank \begin{pNiceArray}{cccc}
    | & | & | & | \\
    \vect{v}_1 & \vect{v}_2 & \vect{v}_3 & \vect{v}_4 \\
    | & | & | & | \\
\end{pNiceArray} = 2,
$$
$\vect{v}_1$, $\vect{v}_2$, $\vect{v}_3$, $\vect{v}_4$ do not span $\R^3$.

\subsection*{Part B}

Since $\vect{v}_1$, $\vect{v}_2$, $\vect{v}_3$, $\vect{v}_4 \in \R^3$ and there are four vectors, at least one of them must be a multiple of each other. As such, they must be linearly dependent. In addition, we found in 3(a) that the matrix comprised of these column vectors is not full rank.

\subsection*{Part C}

$\vect{v}_1$, $\vect{v}_2$, $\vect{v}_3$, $\vect{v}_4$ do not form a basis for $\R^3$ because they are linearly dependent. In 3(a), we found that a matrix comprised of these column vectors is not full rank. Specifically, its rank is $2$. This means that they are only two linearly independent vectors in the set and are therefore not sufficient to span $\R^3$. As such, it is not possible to choose a subset to be a basis of $\R^3$.

\subsection*{Part D}

In 3(a), we found that a matrix comprised of these column vectors has a rank of 2. As such, a basis of subspace spanned by these vectors will contain two vectors. Hence, cardinality of the basis i.e., the \textit{dimension} of $\operatorname{span} \set{\vect{v}_1, \vect{v}_2, \vect{v}_3, \vect{v}_4}$ is 2.

\section*{Question 4}

\subsection*{Parts A through C}

Refer to Section \code{\%\% Question 3} in the \code{math425hw3.m} file for the relevant code.

\end{document}
