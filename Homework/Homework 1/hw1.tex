\documentclass[11pt]{amsart}

\usepackage{amsthm, amssymb,amsmath}
\usepackage{graphicx}

\theoremstyle{definition}  % Heading is bold, text is roman
\newtheorem{theorem}{Theorem}
\newtheorem{definition}{Definition}
\newtheorem{example}{Example}

\newcommand{\ojo}[1]{{\sffamily\bfseries\boldmath[#1]}}

\oddsidemargin 0pt
\evensidemargin 0pt
\marginparwidth 0pt
\marginparsep 10pt
\topmargin -10pt
\headsep 10pt
\textheight 8.4in
\textwidth 7in

%\input{../header}


\begin{document}

%\homework{}{Homework I}


\noindent In this homework, you will write some {\tt MATLAB} code and do computations in {\tt MATLAB}. Include all code and computations
in a {\tt MATLAB} file named {\tt math425hw1.m}. You will need to submit this file along with a document containing your answers which do not
involve {\tt MATLAB}.\\


\noindent
{\bf 1.} In this exercise you will implement Gaussian elimination and backward substitution for solving systems of equations in {\tt MATLAB}. \\
{\bf a)} Create a function called {\tt myGaussianElimination} which takes as input a regular $n \times (n+1)$ matrix $A$. The output is an $n \times (n+1)$
upper-triangular matrix $U$ whose diagonals are nonzero. Recall that if a matrix is regular, then during Gaussian elimination no pivot entry will be zero and
therefore no row swapping will be needed. Your function should issue an error message if it encounters a pivot that is zero. \\
{\bf b)} Create a function called {\tt myBackwardSubstitution} which takes as input an $n \times n$ upper-triangular matrix $U$ with nonzero
diagonal entries and a $n \times 1$ column vector $c$. The output is a $n \times 1$ column vector $x$ which is the solution to $Ux = c$. \\
{\bf c)} Finally create a function {\tt myLinearSolution} that will take as input an $n \times n$ regular matrix $A$ and a $n \times 1$ column vector $b$.
The output is a $n \times 1$ column vector which is the solution to $Ax = b$. Use the two functions you have created in the first two parts. Your function
should issue the appropriate error messages.  \\


\noindent {\bf 2.} In this exercise we will use the following matrix to illustrate how to compute the $LU$ decomposition and how to use it::
$$ A = \left( \begin{array}{rrrr} -8 & -2 & 3 & 1 \\  1 & -2 &  0  & 2 \\ -4 & -1 & 3 & 2 \\ 4 & 1 & -1 & -1 \end{array} \right)$$
By Gaussian elimination one can reduce the above $A$ to an upper triangular matrix by using  row operations of the type $cR_i + R_j \rightarrow R_j$. \\
{\bf a)} State the first row operation you would do in order to reduce $A$ to an upper triangular matrix and apply this row operation to 
obtain the matrix $A_1$.\\
{\bf b)} Find the corresponding elementary matrix $E_1$. Then compute $E_1A$.  What do you observe ? \\
{\bf c)}  State and apply the next row operation that needs to be applied to $A_1$ to obtain $A_2$. \\
{\bf d)}  Construct the corresponding elementary matrix $E_2$.  Compute $E_2 A_1$.  \\
{\bf e)} Now continue to construct $E_3$ and $ E_4$. Verify $E_4E_3E_2E_1A  = U$ where $U$ is an upper triangular matrix. \\
{\bf f)} Compute the inverses of $E_1, \ldots, E_4$. State the pattern you see. \\
{\bf g)} Argue that $A = LU$ where $L= E_1^{-1}E_2^{-1} E_3^{-1}E_4^{-1}$.  What kind of matrix is $L$ ? \\
We have obtained
an LU factorization of $A$.  To check your work use \\
{\tt >> [L, U] = lu(A)} \\
command in MATLAB to get the LU factorization of $A$. \\
{\bf h)} Suppose we want to solve $Ax = b$ where $b = \left( \begin{array}{rrrr} -2 & 6 &  -5 &  1 \end{array} \right)^T$. Show that you can 
do this in two steps once you have the LU factorization of $A$:  first solve $Ly = b$ and then $Ux = y$. Find the solution $x$ using
your function ${\tt myLinearSolution}$ and the LU decomposition. Check you get the same answer. Include your computations in {\tt math425hw1.m}. \\

\noindent {\bf 3.} Suppose in the linear system $Ax = b$ the matrix $A$  is an $n \times n$ regular matrix. \\
{\bf a)} Count the number of multiplication/division operations
needed to find the solution $x$ by Gaussian elimination and backward substitution. The answer should be a formula involving a polynomial in $n$.
What is the degree and the leading coefficient of this polynomial ? \\
{\bf b)} Now count the number of multiplications/divisions to compute $A^{-1}$ and to find $x$ by $x=A^{-1}b$. What is the degree and the leading
coefficient of the resulting polynomial in $n$ in your formula ? \\
{\bf c)} Based on your answers above which method will you use for large $n$ ? Explain. \\

\noindent {\bf 4.} To appreciate the issue of numerical inaccuracy in the linear algebra computations we will look at the problem of 
solving a system $Ax = b$ when $A$ is a square invertible matrix. In that case, we know we can compute the unique solution 
by $ x = A^{-1} b$. However, although a matrix is invertible, it could be almost non-invertible. We call such a matrix {\it ill-conditioned}. 
If we try to compute  $A^{-1}$ of an ill-conditioned matrix numerical rounding errors could accumulate to such a degree that the computed
$A^{-1} b$ may not be close to the true solution. Now go through the following steps and include your computations in your  file {\tt math425hw1.m}.

\vskip 0.1cm
\noindent There is a well-known class of ill-conditioned matrices, called {\it Hilbert matrices}. For each positive integer $n$ there is a Hilbert 
matrix $H_n$. You can see the general form of $H_n$ in your textbook on page 57. \\
{\bf a)} You can generate Hilbert matrices in MATLAB. For instance, 
type \\
{\tt >> hilb(5)} \\
to see $H_5$. Using MATLAB's {\tt inv} command to compute the inverse of a square matrix,  compute the inverse of $H_5$, $H_{10}$, $H_{15}$, etc.
Look at the output carefully. What is the feature of the inverses that jumps at you ?\\
{\bf b)} Now we will solve a system $H_{15}x = b$, by $H_{15}^{-1}b$. First generate a random (column) vector with $15$ entries by  \\
{\tt >> x = rand(15,1) } \\
Then we compute  \\ 
{\tt >> b = hilb(15)*x} \\
Now we know the solution to the system $H_{15}x =b$:  it is the random vector you generated in the first place. As a last step, compute
the solution to this system using $H_{15}^{-1}b$, and compare with the true solution. Look at the MATLAB output carefully. It gives you more information. 

\vskip 0.3cm
\noindent {\bf 5.} Suppose that $A$ is an $n \times n$ matrix that (somehow) you know to be regular. This means that $A$ has an LU decomposition, i.e.,
$A = LU$ where $L$ is a lower-triangular matrix whose diagonal entries are equal $1$ and $U$ is an upper-triangular matrix with nonzero diagonals.
Describe how you can compute the entries of $L$ and $U$ {\bf without} the help of Gaussian elimination. How many multiplications/divisions
do you need? Compare the  answer to  your answers in question 3.
 


\end{document}


