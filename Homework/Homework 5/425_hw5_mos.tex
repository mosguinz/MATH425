\documentclass[12pt]{article}
\usepackage[margin=2.5cm]{geometry}
\usepackage{fancyhdr,latexsym,amssymb,amsmath,graphicx}
\usepackage{algorithm}
\usepackage{cancel}
\usepackage[noend]{algpseudocode}
\usepackage[pdftex]{hyperref}
\pagestyle{fancy}
\usepackage[parfill]{parskip} % Do not indent between empty lines
\usepackage{mathtools}
\usepackage{nicematrix}
\usepackage{xfrac}
\usepackage{bm}
\usepackage{esvect}

% For the \set notation
\usepackage{xparse} 
\DeclarePairedDelimiterX{\set}[1]{\{}{\}}{\setargs{#1}}
\NewDocumentCommand{\setargs}{>{\SplitArgument{1}{;}}m}
{\setargsaux#1}
\NewDocumentCommand{\setargsaux}{mm}
{\IfNoValueTF{#2}{#1} {#1\,\delimsize|\,\mathopen{}#2}}%{#1\:;\:#2}

% Some common symbols
\newcommand{\veq}{\mathrel{\rotatebox{90}{$=$}}}
\newcommand{\vneq}{\mathrel{\rotatebox{90}{$\neq$}}}
\newcommand{\vect}[1]{\vv{\mathbf{#1}}}
\newcommand{\code}[1]{\texttt{#1}}
\newcommand{\R}{\mathbb{R}}
\newcommand{\rank}{\operatorname{rank}}

\lhead{MATH 425}
\chead{Homework 5}
\rhead{Sitthisarnwattanachai}

\title{MATH 425 Homework 5}
\author{mosguinz}
\date{October 2024}

\begin{document}

\section*{Question 1}

Let $A$ be a matrix composed of the column vectors, such that $A = \begin{pmatrix}
    1 & 0 & 1 \\
    2 & 1 & 0 \\
    -1 & -2 & 3 \\
    0 & -1 & 2
\end{pmatrix}$. We first note that the system contains no solution:
$$
\operatorname{rref}\begin{pNiceArray}{c|c}
    A & \vect{b}    
\end{pNiceArray} =\begin{pNiceArray}{ccc|c}
     1 & 0 &  1 & 0 \\
     0 & 1 & -2 & 0 \\
     0 & 0 &  0 & 1 \\
     0 & 0 &  0 & 0 \\
\end{pNiceArray}
$$

Using \verb|A\b|, we find that the least-squares solution is
$
\vect{x}^* = \verb|A\b| = \begin{pmatrix}
    1/4 \\ 0 \\ 1/4
\end{pmatrix}
$. Hence, the closest point from $\vect{b}$ is
$
A\vect{x}^* = \verb|A * (A\b)| = \begin{pmatrix}
    1/2 \\ 1/2 \\ 1/2 \\ 1/2
\end{pmatrix}
$.

\section*{Question 2}

Refer to section \code{\%\% Question 2} in the \code{math425hw5.m} file for the relevant code.

\subsection*{Method one}

This method involves using the $QR$ decomposition. Where $A = QR = \begin{pmatrix}
    1 & 2 & 1 \\
    2 & 1 & 0 \\
    -1 & -2 & 3 \\
    0 & -1 & 2
\end{pmatrix}$, we have that

$$
\begin{array}{ll}
A\vect{x} = \vect{b} \\
QR\vect{x} = \vect{b} \\
\cancel{Q^\top Q}R\vect{x} = Q^\top\vect{b}  \\
R\vect{x} = Q^\top\vect{b}.
\end{array}
$$

Since $Q$ is orthogonal, $Q^\top Q=I$. And since $R$ is upper triangular, we can simply solve for $\vect{x}$ by performing backward substitution. With the $QR$ factorization of $A$ and the magic of MATLAB, the least-squares solution is
$$
\vect{x} = \code{fixed.backwardSubstitute(R, Q' * b)} = \begin{pmatrix}
    -1 \\ 2 \\ 3
\end{pmatrix}.
$$

\subsection*{Method two}

This method involves using the Cholesky decomposition. Since $A^\top A$ is symmetric, Using MATLAB, we can find the Cholesky factorization using \verb|chol(A'A)|. Where $R$ is  an upper triangular matrix such that $R^\top R = A^\top A$, we have that
\begin{align*}
    A^\top A\vect{x} = A^\top\vect{b} &\iff R^\top R\vect{x} = A^\top\vect{b} \\
    &\iff R^\top\vect{y} = A^\top\vect{b}
\end{align*}
where $\vect{y} = R\vect{x}$.

As such, we can compute the least-squares solution in two steps. First, solve the lower triangular system for $\vect{y}$:
$$
\vect{y} = \code{fixed.forwardSubstitute(R', A' * b)} = \begin{pNiceArray}{r}
    -5.4272... \\
    6.0554... \\
    7.6731...
\end{pNiceArray}
$$

Then, solve the upper triangular system, giving us the least-squares solution
$$
\vect{x} = \code{fixed.backwardSubstitute(R, y)} = \begin{pNiceArray}{c}
    -1 \\ 2 \\3
\end{pNiceArray}.
$$

\section*{Question 3}

Refer to section \code{\%\% Question 3} in the \code{math425hw5.m} file for the relevant code.

From the table, we construct a system of equation $A\vect{x}=\vect{b}$, where
$$
\begin{cases}
    \alpha_1 + 1989\beta_1 &= 86.4 \\
    \alpha_2 + 1990\beta_2 &= 89.8 \\
    \alpha_3 + 1991\beta_3 &= 92.8 \\
    &\vdotswithin{=} \\
    \alpha_{11} + 1999\beta_{11} &= 129.5 \\
\end{cases}
\iff
\begin{array}[t]{cccc}
\begin{pmatrix}
    1 & 1989 \\
    1 & 1990 \\
    1 & 1991 \\
    \vdots & \vdots\\
    1 & 1999
\end{pmatrix}
&
\begin{pmatrix}
    \alpha \\ \beta
\end{pmatrix}
&
=
&
\begin{pmatrix}
    86.4 \\
    89.8 \\
    92.8 \\
    \vdots \\
    129.5
\end{pmatrix}
=\vect{b}.
\\
\overset{\veq}{A} & \overset{\veq}{\vect{x}}
\end{array}
$$

Then, applying $A^\top$ to both sides and solving for $\vect{x}$ yields
$$
A^\top A\vect{x}=A^\top\vect{b} \implies
\vect{x}^* = \verb|A' * A \ A' * b| \approx
\begin{pmatrix}
    -54024/7   \\  
     863/220  \\
\end{pmatrix},
$$

producing a line of best fit that estimates the median price (in thousand dollars)
$$
y \approx -\frac{54024}{7}x + \frac{863}{220} \approx -7717.70909x + 3.92273
$$

for the year $x$. Hence, for the years $x=2005$ and $x=2010$, the estimated median price are approximately \$147359 and \$166973, respectively.

\section*{Question 4}

\subsection*{For all parts}

Refer to section \code{\%\% Question 4} in the \code{math425hw5.m} file for the relevant code.

\end{document}
